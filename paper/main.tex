% bare_jrnl_compsoc, texV1.4b, 2015/08/26, Michael Shell
\documentclass[10pt,journal,compsoc]{IEEEtran}

% *** MISC UTILITY PACKAGES ***

\newcommand{\note}[1]{\textcolor{magenta}{#1}} 
\usepackage[nocompress]{cite}
\usepackage{hyperref} % autoref
% *** GRAPHICS RELATED PACKAGES ***
\ifCLASSINFOpdf
   \usepackage[pdftex]{graphicx}
  % declare the path(s) where your graphic files are
   \graphicspath{{./figures/}}
  % and their extensions so you won't have to specify these with
  % every instance of \includegraphics
  % \DeclareGraphicsExtensions{.pdf,.jpeg,.png}
\else
  % or other class option (dvipsone, dvipdf, if not using dvips). graphicx
  % will default to the driver specified in the system graphics.cfg if no
  % driver is specified.
   \usepackage[dvips]{graphicx}
  % declare the path(s) where your graphic files are
   \graphicspath{{./figures/}}
  % and their extensions so you won't have to specify these with
  % every instance of \includegraphics
   \DeclareGraphicsExtensions{.eps}
\fi

% latex, and pdflatex in dvi mode, support graphics in encapsulated
% postscript (.eps) format. pdflatex in pdf mode supports graphics
% in .pdf, .jpeg, .png and .mps (metapost) formats. Users should ensure
% that all non-photo figures use a vector format (.eps, .pdf, .mps) and
% not a bitmapped formats (.jpeg, .png). The IEEE frowns on bitmapped formats
% which can result in "jaggedy"/blurry rendering of lines and letters as
% well as large increases in file sizes.

% *** MATH PACKAGES ***
\usepackage{amsmath}
%% with other math-related packages, you may want to disable it.
\usepackage{amsmath,amsfonts,amssymb,eulervm,xspace}
%\usepackage{mathrsfs} % math script fonts

% *** SPECIALIZED LIST PACKAGES ***
\usepackage{xcolor}
\usepackage{algorithmic}
\usepackage[utf8]{inputenc}
\usepackage{subfig} %ieee does not like subfigure
\usepackage{multicol}
\usepackage{tikz}
\usetikzlibrary{cd} % commutative diagrams
\newtheorem{prop}{Proposition} %math?
\usepackage[switch]{lineno}
\usepackage{minted}
\setminted[python]{fontsize=\scriptsize, 
                   linenos,
                   numbersep=8pt,
                   autogobble, 
                   frame=lines,
                   framesep=3mm} 
% *** ALIGNMENT PACKAGES ***
\usepackage{array}
\usepackage{tabulary}
% IEEEtran contains the IEEEeqnarray family of commands

% *** SUBFIGURE PACKAGES ***
\ifCLASSOPTIONcompsoc
  \usepackage[caption=false,font=footnotesize,labelfont=sf,textfont=sf]{subfig}
\else
  \usepackage[caption=false,font=footnotesize]{subfig}
\fi

% *** FLOAT PACKAGES ***
\usepackage{dblfloatfix}

% *** PDF, URL AND HYPERLINK PACKAGES ***
\usepackage{url}

% *** Do not adjust lengths that control margins, column widths, etc. ***
% *** Do not use packages that alter fonts (such as pslatex).         ***
% There should be no need to do such things with IEEEtran.cls V1.6 and later.
% (Unless specifically asked to do so by the journal or conference you plan
% to submit to, of course. )

\usepackage{notation} %notation conventions
% correct bad hyphenation here
\hyphenation{op-tical net-works semi-conduc-tor}


\begin{document}
%
\title{Topological Equivariant Artist Model for Visualization Library Architecture}
% author names and IEEE memberships
\author{Hannah~Aizenman, Thomas~Caswell, and~Michael~Grossberg,~\IEEEmembership{Member,~IEEE,}% <-this % stops a space
\IEEEcompsocitemizethanks{\IEEEcompsocthanksitem H. Aizenman and M. Grossberg are with the department of Computer Science, City College of New York. 
\protect\\
% note need leading \protect in front of \\ to get a newline within \thanks as
% \\ is fragile and will error, could use \hfil\break instead.
E-mail: haizenman@ccny.cuny.edu, mgrossberg@ccny.cuny.edu 
\IEEEcompsocthanksitem Thomas Caswell is with National Synchrotron Light Source II, Brookhaven National Lab 
\protect \\
E-mail: tcaswell@bnl.gov}% <-this % stops an unwanted space
\thanks{Manuscript received X XX, XXXX; revised X XX, XXXX.}
}


% for Computer Society papers, we must declare the abstract and index terms
% PRIOR to the title within the \IEEEtitleabstractindextext IEEEtran
% command as these need to go into the title area created by \maketitle.
% As a general rule, do not put math, special symbols or citations
% in the abstract or keywords.
\IEEEtitleabstractindextext{%
\begin{abstract}
The abstract goes here.
\end{abstract}

% Note that keywords are not normally used for peerreview papers.
\begin{IEEEkeywords}
%Computer Society, IEEE, IEEEtran, journal, \LaTeX, paper, template.
\end{IEEEkeywords}}


% make the title area
\maketitle


\IEEEpeerreviewmaketitle



\IEEEraisesectionheading{\section{Introduction}\label{sec:introduction}}


\IEEEPARstart{T}his paper uses methods from topology and category theory to develop a model of the transformation from data to graphical representation. This model provides a language to specify how data is structured and how this structure is carried through in the visualization, and serves as the basis for a functional approach to implementing visualization library components. Topology allows us to describe the structure of the data and graphics in a generalizable, scalable, and trackable way. Category theory provides a framework for separating the transformations implemented by visualization libraries from the various stages of visualization and therefore can be used to describe the constraints imposed on the library components \cite{wielsManagementEvolvingSpecifications1998,goguenCategoricalManifesto1991}. Well constrained modular components are inherently functional\cite{hughesWhyFunctionalProgramming1989}, and a functional framework yields a library implementation that is likely to be shorter, clearer, and more suitd to distributed, concurrent, and on demand tasks\cite{huHowFunctionalProgramming2015}. Using this functional approach, this paper contributes a practical framework for decoupling data processing from visualization generation in a way that allows for modular visualization components that are applicable to a variety of data sets in different formats. \note{is it OK that this is something reviewer 4 wrote}



\section{Related Work}
This work aims to develop a model for describing visualization transformations that can serve as guidance for how to architecture a general purpose visualization library. We define a general purpose visualization library as one that provides non domain specific building block components\cite{wongsuphasawatNavigatingWideWorld2021} for building visualizations, for example functions for converting data to color or encoding data as dots. In this section, we describe how visualization libraries attempt this goal and discuss work that formally describes what properties of data should be preserved in a visualization. We restrict the properties of data that should be preserved to 

\begin{LaTeXdescription}
  \item [continuity] how elements in a dataset are connect to each other, e.g. discrete rows in a table, networked nodes, pixels in an image, points on a line
  \item [equivariance] functions on data that have an equivalent effect on the graphical representation, e.g. rotating a matrix has a matching rotation of the image, translating the points on a line has a matching visual shift in the line plot
\end{LaTeXdescription}

\subsection{Continuity}
\begin{figure}[!h]
  \includegraphics[width=\columnwidth]{k_different_types.png}
  \caption{Continuity is how elements in a data set are connected to each other, which is distinct from how the data is structured. The rows in (a) are discrete, therefore they have discrete continuity as illustrated by the discrete dots. The gaussian in (b) is a 1D continuous function, therefore the continuity of the elements of the gaussian can be represented as a line on an interval (0,1). In (c), every element of the globe is connected to its nearest neighbors, which yields a 2D continuous continuity as illustrated by the square.}
  \label{fig:related-work:continuity}
\end{figure}

Continuity is a representation of how the elements in a dataset are connected to each other. For example, in \autoref{fig:related-work:continuity}, each station record in the table is independent of the others; therefore, the continuity of the table is discrete. The data provided by the gaussian are points sampled along the curve, therefore the continuity of the points on the line is 1D continuous. Every point on the globe is connected to its 6 nearest cardinal neighboring points (NW, N, NE, E, SE, S, SW, W). 

\begin{figure}[!h]
  \includegraphics[width=\columnwidth]{whycontinuity.png}
  \caption{Continuity is implicit in choice of visualization rather than explicitly in choice of data container. The line plots in (b) are generated by a 2D table (a). Structurally this table can be identical to the 2D matrix (a) that generates the image in (c).}
  \label{fig:related-work:visual-algorithm}
\end{figure}

Often continuity is expressed in the choice of visual algorithm (visualization type), as explored in taxonomies by Tory and M\"{o}ller \cite{toryRethinkingVisualizationHighlevel2004} and Chi\cite{chiTaxonomyVisualizationTechniques2000}. For example, in \autoref{fig:related-work:visual-algorithm} the same table can be interpreted as a set of 1D continuous curves when visualized as a collection of line plots or as a 2D surface when visualized as an image.  This means that often there is no way to express data continuity independent of visualization type, meaning most visualization libraries will allow, for example, visualizing discrete data as a line plot or an image. General purpose visualization libraries-such as Matplotlib\cite{hunterMatplotlib2DGraphics2007}, Vtk\cite{hanwellVisualizationToolkitVTK2015,geveciVTK2012}, and D3 \cite{bostockDataDrivenDocuments2011}-carry distinct data models as part of the implementation of each visual algorithm. The lack of unified data model means that each plot in a linked\cite{beckerBrushingScatterplots1987,bujaInteractiveData1991} visualization is treated as independent, as are the transforms converting each field in the data to a visual equivalent.

Domain specific libraries can guarantee consistency because they have a single model of the data in their software design, as discussed in Heer and Agarwal \cite{HeerSoftware2006}'s survey of visualization software design patterns. For example, the relational database is core to tools influenced by APT, such as Tableau\cite{StoltePolaris2002,hanrahanVizQL2006,MackinlayShowme2007} and the Grammar of Graphics\cite{wilkinsonGrammarGraphics2005} inspired ggplot\cite{wickhamGgplot2ElegantGraphics2016a}, Vega\cite{satyanarayanDeclarativeInteractionDesign2014} and Altair\cite{vanderplasAltairInteractiveStatistical2018}. Images underpin scientific visualization tools such as Napari\cite{nicholas_sofroniew_2021_4533308} and ImageJ\cite{schneiderNIHImageImageJ2012} and the digital humanities oriented ImagePlot\cite{studiesCulturevisImageplot2021} macro; the need to visualize and manipulate graphs has spawned tools like Gephi\cite{bastianGephiOpenSource2009}, Graphviz\cite{ellsonGraphvizOpenSource2002}, and Networkx\cite{HagbergExploringNetwork2008}. 

 
\subsubsection{Fiber Bundles}
\begin{figure}[h!]
  \includegraphics[width=\columnwidth]{fiberbundle.png}
  \caption{A fiber bundle is mathematical construct that allows us to express the relationship between data and continuity. The \textcolor{total}{total} space \dtotal is the topological space in which the data is embedded. The \textcolor{fiber}{fiber} space \dfiber\ is embedded in \dtotal\ and is the set of all possible values that any
  \note{add big rectangle E}}
  \label{fig:related-work:fiber-bundle}
\end{figure}


A model that allows for expressing data continuity in a general flexible way allows us to express, and therefore be faithful to, the multivariate consistency constraints described by Qu and Hullman\cite{hullmanKeeping2018}. A general data model also allows for consistent adaptation to modern data needs, such as complex, metadata rich, distributed, or streaming data. The mathematical theory of fiber bundles provides one such abstraction that can express complicated dimensionality and continuity without being tied to any one data container type, as proposed by Butler, Bryson, and Pendley\cite{butlerVisualizationModelBased1989,butlerVectorBundleClassesForm1992}. In this paper, we build on their work that proposes using topological spaces to represent different properties of data. Here we present a brief summary of topology, for more information see Hatcher\cite{hatcherAlgebraicTopology2002}, Munkres\cite{munkresElementsAlgebraicTopology1984}, and Bradley et. al. \cite{bradleyTopologyCategoricalApproach2020}. 

A topological space a topological space $(X, \mathscr{T})$ is a set $X$ with a topology $\mathscr{T}$. Topologies are collections of open sets 


such that the empty set and $X$ are in the collection of open sets $\mathscr{T}$, the union of elements in $\mathscr{T}




Specifically, Butler, Bryson, and Pendley suggest that fiber bundles be the basis of an abstract data model. Fiber bundles are a collection $(\dtotal, \dbase, \dfiber, \pi)$ of topological spaces

\begin{LaTeXdescription}
  \item[\textcolor{total}{Total Space} \dtotal]
  \item[\textcolor{fiber}{Fiber Space} \dfiber]
  \item[\textcolor{base}{Base Space} \dbase]  
\end{LaTeXdescription}

with a projection map $\pi:\dtotal\rightarrow\dbase$ that connects every point in \dtotal\ to a point in \dbase. 
\begin{equation}
  \label{eq:fiber_bundle}
  \begin{tikzcd}
      \dfiber \arrow[r, hook] & \dtotal \arrow[r, "\pi"] & \dbase
  \end{tikzcd}
\end{equation}

As indicated by $\hookrightarrow$, the fiber space \dfiber\ is embedded inside the total space \dtotal\. This is illustrated in \autoref{fig:related-work:fiber-bundle}, wherein the data lives in the manifold \dtotal. Each data point has two dimensions, represented as the square fiber \dfiber. While the data is embedded in \dtotal, it's continuity is that each point lies along a line, represented by the base space \dbase. 


\subsection{Sheaf Maps \sheaf}
We can use sheafs to ensure continuity even when the data is broken up. Is the glue rules -> we can haz parallism. 
\begin{equation}
  \vartist: \sheaf(\dtotal) \rightarrow \sheaf(\gtotal)
  \label{eq:math:artist:artist}
\end{equation}

\begin{figure}[h!]
  \includegraphics[width=\columnwidth]{sheaf.png}
  \caption{}
\end{figure}


\subsection{Equivariance}
What is it?

\begin{figure}[!h]
  \includegraphics[width=\columnwidth]{equiv.png}
  \caption{Equivariance is that a transformation on the data has a corresponding transformation in the graphical representation. For example, in this figure the data is scaled by a factor 10. Equivalently the line plot is scaled by factor of 10, resulting in a shrunken line plot. Either a transformation on the data side can induce a transformation on the visual side, or a transformation on the visual side indicates that there is also a transformation on the data side. }
\end{figure}

\begin{tikzcd}
  data \arrow[r] \arrow[d, "function"] & representation \arrow[r] & visual\;stimulus \arrow[d, "visual\;equivalent\;to\;function"] \\
  data \arrow[r]                       & representation \arrow[r] & visual\; stimulus                                          
\end{tikzcd}

\subsubsection{Category Theory}
In this work, we propose that equivariance constraints can be expressed using category theory. Vickers et. al provide a brief introduction to category theory for visualization practitioners \cite{vickersUnderstandingVisualizationFormal2013}, but their work focuses on data, representation, and evocation, while this paper is aims to provide guidance on how the map from data to representation should be implemented. 

\section{Artist} 
%%- brief intro to artist in most simple form $\vartist \dtotal \rightarrow \gtotal$ 
The \textcolor{artist}{Artist $\mathcal{\vartist}$} is a transformation from a 
%% flesh out categorical framing of artist, walk through how eq 16 is part of implementing the artists
%% https://github.com/story645/proposal/blob/main/notes/meetings/2021_08_30.md

%% domain of A is category E, with sheaves on E7   
%% range of A is category H, sheaves on H
%% category bundles, functions are bundles maps, 
%% subset/restriction is a type of bundle map - bundle over subset is bundle over whole thing, which induces map on section which goes other way which is restriction
%% bundle map induces a sheaf map, fibers themselves are categories, map from fiber to fiber that's legit in F
%%% what is the set up? category of bundles, bundle maps - fiber have some structure/any structure/no structure - bundle maps should be functors, A is gonna be a functor, H is a category of bundles + sheaves of bundles, A is a functor
%%% we claim that for visualizations, A decomposes as 
%%% nu bundle map on bundle + functor on fibers, Q takes you to H
%%% K to S is pullback of deform retract, A is functor on imageA in H


\begin{equation}
  \vartist: \dtotal \rightarrow \gtotal
  \label{eq:math:artist:artist}
\end{equation}

\subsection{Data Bundle \dtotal}
\begin{figure}[h!]
  \includegraphics[width=\columnwidth]{fiberbundle.png}
  \caption{
  
  \note{replace with more concrete}}
\end{figure}
We use topology to model 

The The continuity of the data is encoded in the base space \dbase. 

The properties of the variables are encoded in the fiber space \dfiber. The \textcolor{fiber}{fiber} is a topological space 



Spivak provides notation for describing the set of all possible values $U_{\sigma}$ of a single column with name $c$ and type $T$. This set of values is the fiber "F" $F = U_{\sigma}(c)$. When data is multivariate, the fiber F is the cartesian product of each columns fiber space (5)

\subsubsection{Structured Data: Section \dsection}
\begin{equation}
  \begin{tikzcd}
      \dfiber \arrow[r, hook] & \dtotal \arrow[d, "\pi"'] \\
                        & \dbase \arrow[u, "\dsection"', bend right]
  \end{tikzcd}
\end{equation}

\subsubsection{Structure: Continuity and Equivariant Actions}
\begin{equation}
  \begin{tikzcd}
      \dfiber_i \arrow[d, "m_j"'] \arrow[rd, "m_j\;\circ\; m_k"] & \\
      \dfiber_i \arrow[r, "m_k"']  & \dfiber_i
  \end{tikzcd}
\end{equation}


\subsection{Graphic Bundle \gtotal} %probably doesn't all need to be subsections, just figure + short description
\begin{figure}[h!]
  \includegraphics[width=\columnwidth]{render.png}
\end{figure}

\begin{equation}
  \begin{tikzcd}[ampersand replacement=\&]
      \gfiber \arrow[r, hook] \& \gtotal \arrow[d, "\pi"'] \\
                        \& \gbase \arrow[u, "\gsection"', bend right]
  \end{tikzcd}
\end{equation}
\paragraph{Continuity: Base Space \gbase}
\paragraph{Equivariance: Fiber \gfiber}
\paragraph{Structured Data: Visual \gsection}



\subsection{Union of Artists}
\label{sec:artist:union}
\begin{figure}[!h]
\centering
\subfloat[Union of H]{\includegraphics[width=.4\columnwidth]{exploding_artist.png}%
\label{fig_first_case}}
%\hfil
\subfloat[Artist with shared/harmonized \vsection]{\includegraphics[width=.4\columnwidth]{combined_artist.png}%
\label{fig_second_case}}
\caption{Simulation results for the network.}
\label{fig_sim}
\end{figure}


\section{Constraints: Construction and formal properties of Artists}
Identify the constraints, category theory makes it easier to spell out these constraints, this map has to be equivariant
ordered data & data on trees, there is a monoid that acts on them & is not clear that if your map acts on them there will be a nice map on ordered
why category theory? 
- F->V->D
-> F->V is 1:1 equivariance
-> V->D is glyph equivariance

What are the formal properties and constraints on xi, nu, and qhat

nu = any function f:E\rightarrow E, equiv V\rightarrow V
f: F_i\rightarrow F_i, v_i: F_i\right P_i, 

define category of structure preserving transformations on F
  - valid maps F->F
  - user in setting up fibers, sets up what are valid transforms/morphisms
  - nu is a functor

%xi is a deform retract, continuous, surjective 

%Q: \Gamma (V) \Gamma(H) - is defined on functions not on outputs

%Qhat to Q can't break continuity when it goes from Q to Q hat

%% xi is screen data binding,  map of glyph back to ideal platonic mark

%% continuity is in the sheafs -> requiring that we have a sheaf map and that nu induces a sheaf map, nu can also act on individual components

%% nus are a sheaf map, act on indivdual fibers, induce a sheaf map on E to V, 
%% pull sheafs and sections up to intro, Q is also sheaf map on S 
%%% Q is a sheaf map, category of structure preserving maps, Q is a functor at least on the image 

\begin{figure*}[]
  \includegraphics[width=\linewidth]{q.png}
\end{figure*}

\begin{equation}
  \label{eq:math:artist:diagram}
  \begin{tikzcd}
      \dtotal \arrow[r, "\vchannel"] \arrow[rd, "\pi"'] & \vtotal \arrow[d, "\pi"] & \vindex^*\vtotal \arrow[r, "\vmark"] \arrow[d, "\vindex^*\pi"'] \arrow[l, "\vindex^*"'] & \gtotal \arrow[ld, "\pi"] \\
                                            & \dbase                  & \gbase \arrow[l, "\vindex"']                                              &                    
      \end{tikzcd}
\end{equation}

\subsection{Graphic to Data: \vindex}
\begin{equation}
  \begin{tikzcd}
      \dtotal \arrow[d, "\pi"'] & \gtotal \arrow[d, "\pi"'] \\
      \dbase                   & \gbase \arrow[l, "\vindex"']
  \end{tikzcd}
  \label{eq:math:graphic:vindex}
\end{equation}


\subsection{Visual Bundle \vtotal}
\begin{equation}
  \begin{tikzcd}[ampersand replacement=\&]
      \vfiber \arrow[r, hook] \& \vtotal \arrow[d, "\pi"'] \\
                        \& \dbase \arrow[u, "\vsection"', bend right]
  \end{tikzcd}
\end{equation}

\subsection{Data to Visual Encodings: \vchannel} %look at what K&S call these stages
Visual Channel Encodings
\begin{figure}[!h]
  \centering
  \subfloat[Artists with shared $\mu_i$ renderered correctly]{\includegraphics[width=1.2in]{partial_fixed.png}%
  \label{fig_first_case}}
  %\hfil
  \subfloat[Artists without shared $\mu_i$]{\includegraphics[width=1.2in]{partial_invalid.png}%
  \label{fig_second_case}}
  \caption{Simulation results for the network.}
  \label{fig_sim}
  \end{figure}

\begin{equation}
  \label{eq:math:artist:nu}
  \{\vchannel_{0}, \ldots, \vchannel_{n}\}: \{\dsection_{0}, \ldots, \dsection_{n}\} \mapsto \{\vsection_{0}, \ldots, \vsection_{n}\}
\end{equation}

We enforce the equivariance constraint

\begin{equation}
  \label{eq:math:artist:nu_commute}
\begin{tikzcd}
  \dtotal_i \arrow[r] \arrow[r, "\vchannel_i"] \arrow[d, "m_{\delement}"'] & \vtotal_i \arrow[d, "m_{\velement}"] \\
  \dtotal_i \arrow[r, "\vchannel_i"]                           & \vtotal_i               
\end{tikzcd}
\end{equation}

\subsubsection{Visual to Graphic: \vmark} % look at what K&S call these stages
Visual encodings to something like marks 
\begin{figure}[!h]
  \includegraphics[width=\columnwidth]{diff_type_q.png}
  \caption{rework this as a commutative box w/ the r in E row associated w/ this qhat(k)}
\end{figure}
 


\section{Case Study}
\begin{figure}[h!]
  \includegraphics[width=\columnwidth]{path_of_q.png}
    \caption{\note{add in xi!}}
  \label{fig:api}
\end{figure}
We implement the \textbf{arrows} in \autoref{fig:api}. \mintinline{python}{axesArtist} is a parent artist that acts as a screen. This allows for the composition described in \autoref{sec:artist:union}

\subsection{\vartist}
\begin{minted}{python}
for local_tau in axesArtist.artist.data.query(screen_bounds, dpi):
    mu = axesArtist.artist.graphic.mu(local_tau)
    rho = axesArtist.artist.graphic.qhat(**mu)
    H = rho(renderer)
\end{minted}

where the artist is already parameterized with the \vindex\ functions and which fibers they are associated to:

\begin{min


\subsubsection{\vindex}
\subsubsection{\vchannel}
\subsubsection{\vmarkd}



\section{Discussion}
\subsection{Limitations}
\subsection{future work}

\section{Conclusion}
The conclusion goes here.


\appendices
\section{Rendering: \gsection}
\section{Manufacturing $\vmarkd \leftarrow \vmark$}
\begin{equation}
  \begin{tikzcd}
      \textcolor{gray!50}{\dtotal} \arrow[r, "\vchannel", color=gray!50] \arrow[rd, "\pi"', color=gray!50] & \vtotal \arrow[d, "\pi"']                  & \vindex^*\vtotal \arrow[r, "\vmark", color=gray!50] \arrow[d, "\vindex^*\pi"'] \arrow[l,  "\vindex^*"'] & \textcolor{gray!50}{\gtotal} \arrow[ld, "\pi", color=gray!50] \\ & \dbase \arrow[u, "\vsection"', bend right] & \gbase \arrow[l, "\vindex"'] \arrow[u, "\vindex^*\vsection"', bend right]               &                          
      \end{tikzcd}
      \label{eq:math:artist:qhat}
\end{equation}



% use section* for acknowledgment
\ifCLASSOPTIONcompsoc
  % The Computer Society usually uses the plural form
  \section*{Acknowledgments}
\else
  % regular IEEE prefers the singular form
  \section*{Acknowledgment}
\fi


The authors would like to thank...


% Can use something like this to put references on a page
% by themselves when using endfloat and the captionsoff option.
\ifCLASSOPTIONcaptionsoff
  \newpage
\fi

% trigger a \newpage just before the given reference
% number - used to balance the columns on the last page
% adjust value as needed - may need to be readjusted if
% the document is modified later
%\IEEEtriggeratref{8}
% The "triggered" command can be changed if desired:
%\IEEEtriggercmd{\enlargethispage{-5in}}

% references section

% can use a bibliography generated by BibTeX as a .bbl file
% BibTeX documentation can be easily obtained at:
% http://mirror.ctan.org/biblio/bibtex/contrib/doc/
% The IEEEtran BibTeX style support page is at:
% http://www.michaelshell.org/tex/ieeetran/bibtex/
\bibliographystyle{IEEEtran}
% argument is your BibTeX string definitions and bibliography database(s)
\bibliography{bibliography}

% biography section 
% If you have an EPS/PDF photo (graphicx package needed) extra braces are
% needed around the contents of the optional argument to biography to prevent
% the LaTeX parser from getting confused when it sees the complicated
% \includegraphics command within an optional argument. (You could create
% your own custom macro containing the \includegraphics command to make things
% simpler here.)
%\begin{IEEEbiography}[{\includegraphics[width=1in,height=1.25in,clip,keepaspectratio]{mshell}}]{Michael Shell}
% or if you just want to reserve a space for a photo:

%\begin{IEEEbiography}{Michael Shell}
%\end{IEEEbiography}

% if you will not have a photo at all:
\begin{IEEEbiographynophoto}{Hannah Aizenman}
Biography text here.
\end{IEEEbiographynophoto}

\begin{IEEEbiographynophoto}{Thomas Caswell}
  Biography text here.
\end{IEEEbiographynophoto}
% insert where needed to balance the two columns on the last page with
% biographies
%\newpage

\begin{IEEEbiographynophoto}{Michael Grossberg}
Biography text here.
\end{IEEEbiographynophoto}

% You can push biographies down or up by placing
% a \vfill before or after them. The appropriate
% use of \vfill depends on what kind of text is
% on the last page and whether or not the columns
% are being equalized.

%\vfill

% Can be used to pull up biographies so that the bottom of the last one
% is flush with the other column.
%\enlargethispage{-5in}

% that's all folks
\end{document}


