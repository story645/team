% bare_jrnl_compsoc, texV1.4b, 2015/08/26, Michael Shell
\documentclass[10pt,journal,compsoc]{IEEEtran}

% *** MISC UTILITY PACKAGES ***

\newcommand{\note}[1]{\textcolor{magenta}{#1}} 
\usepackage[nocompress]{cite}
% *** GRAPHICS RELATED PACKAGES ***
\ifCLASSINFOpdf
   \usepackage[pdftex]{graphicx}
  % declare the path(s) where your graphic files are
   \graphicspath{{./figures/}}
  % and their extensions so you won't have to specify these with
  % every instance of \includegraphics
  % \DeclareGraphicsExtensions{.pdf,.jpeg,.png}
\else
  % or other class option (dvipsone, dvipdf, if not using dvips). graphicx
  % will default to the driver specified in the system graphics.cfg if no
  % driver is specified.
   \usepackage[dvips]{graphicx}
  % declare the path(s) where your graphic files are
   \graphicspath{{./figures/}}
  % and their extensions so you won't have to specify these with
  % every instance of \includegraphics
   \DeclareGraphicsExtensions{.eps}
\fi

% latex, and pdflatex in dvi mode, support graphics in encapsulated
% postscript (.eps) format. pdflatex in pdf mode supports graphics
% in .pdf, .jpeg, .png and .mps (metapost) formats. Users should ensure
% that all non-photo figures use a vector format (.eps, .pdf, .mps) and
% not a bitmapped formats (.jpeg, .png). The IEEE frowns on bitmapped formats
% which can result in "jaggedy"/blurry rendering of lines and letters as
% well as large increases in file sizes.

% *** MATH PACKAGES ***
\usepackage{amsmath}
%% with other math-related packages, you may want to disable it.
\usepackage{amsmath,amsfonts,amssymb,eulervm,xspace}
%\usepackage{mathrsfs} % math script fonts

% *** SPECIALIZED LIST PACKAGES ***
\usepackage{xcolor}
\usepackage{algorithmic}
\usepackage[utf8]{inputenc}
\usepackage{subfig} %ieee does not like subfigure
\usepackage{multicol}
\usepackage{tikz}
\usetikzlibrary{cd} % commutative diagrams
\newtheorem{prop}{Proposition} %math?
\usepackage[switch]{lineno}
\usepackage{minted}
\setminted[python]{fontsize=\scriptsize, 
                   linenos,
                   numbersep=8pt,
                   autogobble, 
                   frame=lines,
                   framesep=3mm} 
% *** ALIGNMENT PACKAGES ***
\usepackage{array}
\usepackage{tabulary}
% IEEEtran contains the IEEEeqnarray family of commands

% *** SUBFIGURE PACKAGES ***
\ifCLASSOPTIONcompsoc
  \usepackage[caption=false,font=footnotesize,labelfont=sf,textfont=sf]{subfig}
\else
  \usepackage[caption=false,font=footnotesize]{subfig}
\fi

% *** FLOAT PACKAGES ***
\usepackage{dblfloatfix}

% *** PDF, URL AND HYPERLINK PACKAGES ***
\usepackage{url}

% *** Do not adjust lengths that control margins, column widths, etc. ***
% *** Do not use packages that alter fonts (such as pslatex).         ***
% There should be no need to do such things with IEEEtran.cls V1.6 and later.
% (Unless specifically asked to do so by the journal or conference you plan
% to submit to, of course. )

\usepackage{notation} %notation conventions
% correct bad hyphenation here
\hyphenation{op-tical net-works semi-conduc-tor}


\begin{document}
%
\title{Topological Equivariant Artist Model for Visualization Library Architecture}
% author names and IEEE memberships
\author{Hannah~Aizenman, Thomas~Caswell, and~Michael~Grossberg,~\IEEEmembership{Member,~IEEE,}% <-this % stops a space
\IEEEcompsocitemizethanks{\IEEEcompsocthanksitem H. Aizenman and M. Grossberg are with the department of Computer Science, City College of New York. 
\protect\\
% note need leading \protect in front of \\ to get a newline within \thanks as
% \\ is fragile and will error, could use \hfil\break instead.
E-mail: haizenman@ccny.cuny.edu, mgrossberg@ccny.cuny.edu 
\IEEEcompsocthanksitem Thomas Caswell is with National Synchrotron Light Source II, Brookhaven National Lab 
\protect \\
E-mail: tcaswell@bnl.gov}% <-this % stops an unwanted space
\thanks{Manuscript received X XX, XXXX; revised X XX, XXXX.}
}


% for Computer Society papers, we must declare the abstract and index terms
% PRIOR to the title within the \IEEEtitleabstractindextext IEEEtran
% command as these need to go into the title area created by \maketitle.
% As a general rule, do not put math, special symbols or citations
% in the abstract or keywords.
\IEEEtitleabstractindextext{%
\begin{abstract}
The abstract goes here.
\end{abstract}

% Note that keywords are not normally used for peerreview papers.
\begin{IEEEkeywords}
%Computer Society, IEEE, IEEEtran, journal, \LaTeX, paper, template.
\end{IEEEkeywords}}


% make the title area
\maketitle


\IEEEpeerreviewmaketitle



\IEEEraisesectionheading{\section{Introduction}\label{sec:introduction}}


\IEEEPARstart{T}his paper uses methods from topology and category theory to develop a model of the transformation from data to graphical representation. This model provides a language to specify how data is structured and how this structure is carried through in the visualization, and serves as the basis for a functional approach to implementing visualization library components. Topology allows us to describe the structure of the data and graphics in a generalizable, scalable, and trackable way. Category theory provides a framework for separating the transformations implemented by visualization libraries from the various stages of visualization and therefore can be used to describe the constraints imposed on the library components \cite{wielsManagementEvolvingSpecifications1998,goguenCategoricalManifesto1991}. Well constrained modular components are inherently functional\cite{hughesWhyFunctionalProgramming1989}, and a functional framework yields a library implementation that is likely to be shorter, clearer, and more suitd to distributed, concurrent, and on demand tasks\cite{huHowFunctionalProgramming2015}. Using this functional approach, this paper contributes a practical framework for decoupling data processing from visualization generation in a way that allows for modular visualization components that are applicable to a variety of data sets in different formats. \note{is it OK that this is something reviewer 4 wrote}



\section{Related Work}
This work aims to develop a model for describing visualization transformations that can be translated into visualization library architecture. In doing so, we describe how visualization libraries attempt this goal and  build on work that formally describes what properties of data should be preserved. We restrict the properties of data that should be preserved to 

\begin{LaTeXdescription}
  \item [continuity] how elements in a dataset are connect to each other, e.g. discrete rows in a table, networked nodes, pixels in an image, points on a line
  \item [equivariance] functions on data that have an equivalent effect on the graphical representation, e.g. rotating a matrix has a matching rotation of the image, translating the points on a line has a matching visual shift in the line plot
\end{LaTeXdescription}

\subsection{Continuity}
\begin{figure}[!h]
  \includegraphics[width=\columnwidth]{k_different_types.png}
  \caption{Continuity is how elements in a data set are connected to each other, which is distinct from how the data is structured. The rows in (a) are discrete, therefore they have discrete continuity as illustrated by the discrete dots. The gaussian in (b) is a 1D continuous function, therefore the continuity of the elements of the gaussian can be represented as a line on an interval (0,1). In (c), every element of the globe is connected to its nearest neighbors, which yields a 2D continuous continuity as illustrated by the square.}
\end{figure}

Continuity is 

We care about continuity b/c 
\begin{figure}[!h]
  \includegraphics[width=\columnwidth]{whycontinuity.png}
  \caption{Continuity is implicit in choice of visualization rather than explicitely in choice of data container. The line plots in (b) are generated by a 2D table (a). Structurally this table can be identical to the 2D matrix (a) that generates the image in (c).}
\end{figure}


We express continuity using fiber bundles, which are ...
\cite{butlerVectorBundleClassesForm1992,butlerVisualizationModelBased1989}
\begin{equation}
  \label{eq:fiber_bundle}
  \begin{tikzcd}
      \dfiber \arrow[r, hook] & \dtotal \arrow[r, "\pi"] & \dbase
  \end{tikzcd}
\end{equation}



\subsection{Equivariance}
What is it?

\begin{figure}[!h]
  \includegraphics[width=\columnwidth]{equiv.png}
  \caption{Equivariance is that a transformation on the data has a corresponding transformation in the graphical representation. For example, in this figure the data is scaled by a factor 10. Equivalently the line plot is scaled by factor of 10, resulting in a shrunken line plot. Either a transformation on the data side can induce a transformation on the visual side, or a transformation on the visual side indicates that there is also a transformation on the data side. }
\end{figure}

\begin{tikzcd}
  data \arrow[r] \arrow[d, "function"] & representation \arrow[r] & visual\;stimulus \arrow[d, "visual\;equivalent\;to\;function"] \\
  data \arrow[r]                       & representation \arrow[r] & visual\; stimulus                                          
\end{tikzcd}

\section{Modeling Visualization Stages: Fiber bundles}

%%- brief intro to artist in most simple form $\vartist \dtotal \rightarrow \gtotal$ 
The \textcolor{artist}{Artist $\mathcal{\vartist}$} is a transformation from 


\subsection{Data Bundle \dtotal}
\begin{figure}[h!]
  \includegraphics[width=2.5in]{k_qspace.png}
  \caption{this is gonna be replaced w/ more concrete}
\end{figure}
We use topology to model 

The The continuity of the data is encoded in the base space \dbase. 

The properties of the variables are encoded in the fiber space \dfiber. The \textcolor{fiber}{fiber} is a topological space 



Spivak provides notation for describing the set of all possible values $U_{\sigma}$ of a single column with name $c$ and type $T$. This set of values is the fiber "F" $F = U_{\sigma}(c)$. When data is multivariate, the fiber F is the cartesian product of each columns fiber space (5)

\paragraph{Structured Data: Section \dsection}
\begin{equation}
  \begin{tikzcd}
      \dfiber \arrow[r, hook] & \dtotal \arrow[d, "\pi"'] \\
                        & \dbase \arrow[u, "\dsection"', bend right]
  \end{tikzcd}
\end{equation}

\paragraph{Structure: Continuity and Equivariant Actions}
\begin{equation}
  \begin{tikzcd}
      \dfiber_i \arrow[d, "m_j"'] \arrow[rd, "m_j\;\circ\; m_k"] & \\
      \dfiber_i \arrow[r, "m_k"']  & \dfiber_i
  \end{tikzcd}
\end{equation}

\subsubsection{Continuity of Data: sheaf \sheaf}
\begin{figure}[h!]
  \includegraphics[width=2.5in]{sheaf.png}
  \caption{}
\end{figure}



\subsection{Graphic Bundle \gtotal} %probably doesn't all need to be subsections, just figure + short description
\begin{figure}[h!]
  \includegraphics[width=\columnwidth]{render.png}
\end{figure}
\begin{equation}
  \begin{tikzcd}[ampersand replacement=\&]
      \gfiber \arrow[r, hook] \& \gtotal \arrow[d, "\pi"'] \\
                        \& \gbase \arrow[u, "\gsection"', bend right]
  \end{tikzcd}
\end{equation}
\paragraph{Continuity: Base Space \gbase}
\paragraph{Equivariance: Fiber \gfiber}
\paragraph{Structured Data: Visual \gsection}


\subsection{Artist}
%% flesh out categorical framing of artist, walk through how eq 16 is part of implementing the artists
%% https://github.com/story645/proposal/blob/main/notes/meetings/2021_08_30.md
\begin{equation}
  \vartist: \sheaf(\dtotal) \rightarrow \sheaf(\gtotal)
  \label{eq:math:artist:artist}
\end{equation}

\section{Union of Artists}
\begin{figure}[!h]
\centering
\subfloat[Artists with shared $\mu_i$ renderered correctly]{\includegraphics[width=1.2in]{combined_artist.png}%
\label{fig_first_case}}
%\hfil
\subfloat[Artists without shared $\mu_i$]{\includegraphics[width=1.2in]{exploding_artist.png}%
\label{fig_second_case}}
\caption{Simulation results for the network.}
\label{fig_sim}
\end{figure}


\section{Construction and formal properties of Artists}

\begin{figure*}[]
  \includegraphics[width=\linewidth]{q.png}
\end{figure*}

\begin{equation}
  \label{eq:math:artist:diagram}
  \begin{tikzcd}
      \dtotal \arrow[r, "\vchannel"] \arrow[rd, "\pi"'] & \vtotal \arrow[d, "\pi"] & \vindex^*\vtotal \arrow[r, "\vmark"] \arrow[d, "\vindex^*\pi"'] \arrow[l, "\vindex^*"'] & \gtotal \arrow[ld, "\pi"] \\
                                            & \dbase                  & \gbase \arrow[l, "\vindex"']                                              &                    
      \end{tikzcd}
\end{equation}

\subsection{Graphic to Data: \vindex}
\begin{equation}
  \begin{tikzcd}
      \dtotal \arrow[d, "\pi"'] & \gtotal \arrow[d, "\pi"'] \\
      \dbase                   & \gbase \arrow[l, "\vindex"']
  \end{tikzcd}
  \label{eq:math:graphic:vindex}
\end{equation}


\subsection{Visual Bundle \vtotal}
\begin{equation}
  \begin{tikzcd}[ampersand replacement=\&]
      \vfiber \arrow[r, hook] \& \vtotal \arrow[d, "\pi"'] \\
                        \& \dbase \arrow[u, "\vsection"', bend right]
  \end{tikzcd}
\end{equation}

\subsection{Data to Representation: \vchannel} %look at what K&S call these stages
\begin{figure}[!h]
  \centering
  \subfloat[Artists with shared $\mu_i$ renderered correctly]{\includegraphics[width=1.2in]{partial_fixed.png}%
  \label{fig_first_case}}
  %\hfil
  \subfloat[Artists without shared $\mu_i$]{\includegraphics[width=1.2in]{partial_invalid.png}%
  \label{fig_second_case}}
  \caption{Simulation results for the network.}
  \label{fig_sim}
  \end{figure}

\begin{equation}
  \label{eq:math:artist:nu}
  \{\vchannel_{0}, \ldots, \vchannel_{n}\}: \{\dsection_{0}, \ldots, \dsection_{n}\} \mapsto \{\vsection_{0}, \ldots, \vsection_{n}\}
\end{equation}

We enforce the equivariance constraint
\begin{equation}
  \label{eq:math:artist:nu_commute}
\begin{tikzcd}
  \dtotal_i \arrow[r] \arrow[r, "\vchannel_i"] \arrow[d, "m_{\delement}"'] & \vtotal_i \arrow[d, "m_{\velement}"] \\
  \dtotal_i \arrow[r, "\vchannel_i"]                           & \vtotal_i               
\end{tikzcd}
\end{equation}

\subsubsection{Representation to Visual: \vmark} % look at what K&S call these stages

\begin{figure}[!h]
  \includegraphics[width=2.5in]{diff_type_q.png}
  \caption{rework this as a commutative box w/ the r in E row associated w/ this qhat(k)}
\end{figure}
 


\section{Case Study}
\begin{figure}[h!]
  \includegraphics[width=\columnwidth]{path_of_q.png}
    \caption{add in xi!}
  \label{fig:api}
\end{figure}
We implement the \textbf{arrows} in \autoref{fig:api} 


\begin{minted}[]{python}
  
\end{minted}

\section{Discussion}
\subsection{Limitations}
\subsection{future work}

\section{Conclusion}
The conclusion goes here.


\appendices
\section{Rendering: \gsection}
\section{Manufacturing $\vmarkd \leftarrow \vmark$}
\begin{equation}
  \begin{tikzcd}
      \textcolor{gray!50}{\dtotal} \arrow[r, "\vchannel", color=gray!50] \arrow[rd, "\pi"', color=gray!50] & \vtotal \arrow[d, "\pi"']                  & \vindex^*\vtotal \arrow[r, "\vmark", color=gray!50] \arrow[d, "\vindex^*\pi"'] \arrow[l,  "\vindex^*"'] & \textcolor{gray!50}{\gtotal} \arrow[ld, "\pi", color=gray!50] \\ & \dbase \arrow[u, "\vsection"', bend right] & \gbase \arrow[l, "\vindex"'] \arrow[u, "\vindex^*\vsection"', bend right]               &                          
      \end{tikzcd}
      \label{eq:math:artist:qhat}
\end{equation}



% use section* for acknowledgment
\ifCLASSOPTIONcompsoc
  % The Computer Society usually uses the plural form
  \section*{Acknowledgments}
\else
  % regular IEEE prefers the singular form
  \section*{Acknowledgment}
\fi


The authors would like to thank...


% Can use something like this to put references on a page
% by themselves when using endfloat and the captionsoff option.
\ifCLASSOPTIONcaptionsoff
  \newpage
\fi

% trigger a \newpage just before the given reference
% number - used to balance the columns on the last page
% adjust value as needed - may need to be readjusted if
% the document is modified later
%\IEEEtriggeratref{8}
% The "triggered" command can be changed if desired:
%\IEEEtriggercmd{\enlargethispage{-5in}}

% references section

% can use a bibliography generated by BibTeX as a .bbl file
% BibTeX documentation can be easily obtained at:
% http://mirror.ctan.org/biblio/bibtex/contrib/doc/
% The IEEEtran BibTeX style support page is at:
% http://www.michaelshell.org/tex/ieeetran/bibtex/
\bibliographystyle{IEEEtran}
% argument is your BibTeX string definitions and bibliography database(s)
\bibliography{bibliography}


% biography section 
% If you have an EPS/PDF photo (graphicx package needed) extra braces are
% needed around the contents of the optional argument to biography to prevent
% the LaTeX parser from getting confused when it sees the complicated
% \includegraphics command within an optional argument. (You could create
% your own custom macro containing the \includegraphics command to make things
% simpler here.)
%\begin{IEEEbiography}[{\includegraphics[width=1in,height=1.25in,clip,keepaspectratio]{mshell}}]{Michael Shell}
% or if you just want to reserve a space for a photo:

%\begin{IEEEbiography}{Michael Shell}
%\end{IEEEbiography}

% if you will not have a photo at all:
\begin{IEEEbiographynophoto}{Hannah Aizenman}
Biography text here.
\end{IEEEbiographynophoto}

\begin{IEEEbiographynophoto}{Thomas Caswell}
  Biography text here.
\end{IEEEbiographynophoto}
% insert where needed to balance the two columns on the last page with
% biographies
%\newpage

\begin{IEEEbiographynophoto}{Michael Grossberg}
Biography text here.
\end{IEEEbiographynophoto}

% You can push biographies down or up by placing
% a \vfill before or after them. The appropriate
% use of \vfill depends on what kind of text is
% on the last page and whether or not the columns
% are being equalized.

%\vfill

% Can be used to pull up biographies so that the bottom of the last one
% is flush with the other column.
%\enlargethispage{-5in}

% that's all folks
\end{document}


