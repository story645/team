% bare_jrnl_compsoc, texV1.4b, 2015/08/26, Michael Shell
\documentclass[10pt,journal,compsoc]{IEEEtran}

% *** MISC UTILITY PACKAGES ***

\newcommand{\note}[1]{\textcolor{magenta}{#1}} 
\usepackage[nocompress]{cite}
\usepackage{hyperref} % autoref
% *** GRAPHICS RELATED PACKAGES ***
\ifCLASSINFOpdf
   \usepackage[pdftex]{graphicx}
  % declare the path(s) where your graphic files are
   \graphicspath{{./figures/}}
  % and their extensions so you won't have to specify these with
  % every instance of \includegraphics
  % \DeclareGraphicsExtensions{.pdf,.jpeg,.png}
\else
  % or other class option (dvipsone, dvipdf, if not using dvips). graphicx
  % will default to the driver specified in the system graphics.cfg if no
  % driver is specified.
   \usepackage[dvips]{graphicx}
  % declare the path(s) where your graphic files are
   \graphicspath{{./figures/}}
  % and their extensions so you won't have to specify these with
  % every instance of \includegraphics
   \DeclareGraphicsExtensions{.eps}
\fi

% latex, and pdflatex in dvi mode, support graphics in encapsulated
% postscript (.eps) format. pdflatex in pdf mode supports graphics
% in .pdf, .jpeg, .png and .mps (metapost) formats. Users should ensure
% that all non-photo figures use a vector format (.eps, .pdf, .mps) and
% not a bitmapped formats (.jpeg, .png). The IEEE frowns on bitmapped formats
% which can result in "jaggedy"/blurry rendering of lines and letters as
% well as large increases in file sizes.

% *** MATH PACKAGES ***
%% with other math-related packages, you may want to disable it.
\usepackage{amsmath, amsthm, amsfonts,amssymb,eulervm,xspace, mathtools}
\usepackage{mathrsfs} % math script fonts
\theoremstyle{definition}
\newtheorem{definition}{Definition}[section]
\theoremstyle{remark}
\newtheorem{example}{Example}[section]
% *** SPECIALIZED LIST PACKAGES ***
\usepackage{xcolor}
\usepackage{algorithmic}
\usepackage[utf8]{inputenc}
\usepackage{subfig} %ieee does not like subfigure
\usepackage{multicol}
\usepackage{tikz}
\usetikzlibrary{cd} % commutative diagrams
\newtheorem{prop}{Proposition} %math?
\usepackage[switch]{lineno}
\usepackage{minted}
\setminted[python]{fontsize=\scriptsize, 
                   linenos,
                   numbersep=8pt,
                   autogobble, 
                   frame=lines,
                   framesep=3mm} 
% *** ALIGNMENT PACKAGES ***
\usepackage{array}
\usepackage{tabulary}
% IEEEtran contains the IEEEeqnarray family of commands

% *** SUBFIGURE PACKAGES ***
\ifCLASSOPTIONcompsoc
  \usepackage[caption=false,font=footnotesize,labelfont=sf,textfont=sf]{subfig}
\else
  \usepackage[caption=false,font=footnotesize]{subfig}
\fi

% *** FLOAT PACKAGES ***
\usepackage{dblfloatfix}

% *** PDF, URL AND HYPERLINK PACKAGES ***
\usepackage{url}

% *** Do not adjust lengths that control margins, column widths, etc. ***
% *** Do not use packages that alter fonts (such as pslatex).         ***
% There should be no need to do such things with IEEEtran.cls V1.6 and later.
% (Unless specifically asked to do so by the journal or conference you plan
% to submit to, of course. )

\usepackage{notation} %notation conventions
% correct bad hyphenation here
\hyphenation{op-tical net-works semi-conduc-tor}


\begin{document}
%
\title{Topological Equivariant Artist Model for Visualization Library Architecture}
% author names and IEEE memberships
\author{Hannah~Aizenman, Thomas~Caswell, and~Michael~Grossberg,~\IEEEmembership{Member,~IEEE,}% <-this % stops a space
\IEEEcompsocitemizethanks{\IEEEcompsocthanksitem H. Aizenman and M. Grossberg are with the department of Computer Science, City College of New York. 
\protect\\
% note need leading \protect in front of \\ to get a newline within \thanks as
% \\ is fragile and will error, could use \hfil\break instead.
E-mail: haizenman@ccny.cuny.edu, mgrossberg@ccny.cuny.edu 
\IEEEcompsocthanksitem Thomas Caswell is with National Synchrotron Light Source II, Brookhaven National Lab 
\protect \\
E-mail: tcaswell@bnl.gov}% <-this % stops an unwanted space
\thanks{Manuscript received X XX, XXXX; revised X XX, XXXX.}
}


% for Computer Society papers, we must declare the abstract and index terms
% PRIOR to the title within the \IEEEtitleabstractindextext IEEEtran
% command as these need to go into the title area created by \maketitle.
% As a general rule, do not put math, special symbols or citations
% in the abstract or keywords.
\IEEEtitleabstractindextext{%
\begin{abstract}
The abstract goes here.
\end{abstract}

% Note that keywords are not normally used for peerreview papers.
\begin{IEEEkeywords}
%Computer Society, IEEE, IEEEtran, journal, \LaTeX, paper, template.
\end{IEEEkeywords}}


% make the title area
\maketitle


\IEEEpeerreviewmaketitle



\IEEEraisesectionheading{\section{Introduction}\label{sec:introduction}}


\IEEEPARstart{T}his paper uses methods from topology and category theory to develop a model of the transformation from data to graphical representation. This model provides a language to specify how data is structured and how this structure is carried through in the visualization, and serves as the basis for a functional approach to implementing visualization library components. Topology allows us to describe the structure of the data and graphics in a generalizable, scalable, and trackable way. Category theory provides a framework for separating the transformations implemented by visualization libraries from the various stages of visualization and therefore can be used to describe the constraints imposed on the library components \cite{wielsManagementEvolvingSpecifications1998,goguenCategoricalManifesto1991}. Well constrained modular components are inherently functional\cite{hughesWhyFunctionalProgramming1989}, and a functional framework yields a library implementation that is likely to be shorter, clearer, and more suitd to distributed, concurrent, and on demand tasks\cite{huHowFunctionalProgramming2015}. Using this functional approach, this paper contributes a practical framework for decoupling data processing from visualization generation in a way that allows for modular visualization components that are applicable to a variety of data sets in different formats. \note{is it OK that this is something reviewer 4 wrote}



\section{Related Work}
This work aims to develop a model for describing visualization transformations that can serve as guidance for how to architecture a general purpose visualization library. We define a general purpose visualization library as one that provides non domain specific building block components\cite{wongsuphasawatNavigatingWideWorld2021} for building visualizations, for example functions for converting data to color or encoding data as dots. In this section, we describe how visualization libraries attempt this goal and discuss work that formally describes what properties of data should be preserved in a visualization. We restrict the properties of data that should be preserved to 

\begin{LaTeXdescription}
  \item [continuity] how elements in a dataset are connect to each other, e.g. discrete rows in a table, networked nodes, pixels in an image, points on a line
  \item [equivariance] functions on data that have an equivalent effect on the graphical representation, e.g. rotating a matrix has a matching rotation of the image, translating the points on a line has a matching visual shift in the line plot
\end{LaTeXdescription}

\subsection{Continuity}
\begin{figure}[!h]
  \includegraphics[width=\columnwidth]{k_different_types.png}
  \caption{Continuity is how elements in a data set are connected to each other, which is distinct from how the data is structured. The rows in (a) are discrete, therefore they have discrete continuity as illustrated by the discrete dots. The gaussian in (b) is a 1D continuous function, therefore the continuity of the elements of the gaussian can be represented as a line on an interval (0,1). In (c), every element of the globe is connected to its nearest neighbors, which yields a 2D continuous continuity as illustrated by the square.}
  \label{fig:related-work:continuity}
\end{figure}

Continuity is a representation of how the elements in a dataset are connected to each other. For example, in \autoref{fig:related-work:continuity}, each station record in the table is independent of the others; therefore, the continuity of the table is discrete. The data provided by the gaussian are points sampled along the curve, therefore the continuity of the points on the line is 1D continuous. Every point on the globe is connected to its 6 nearest cardinal neighboring points (NW, N, NE, E, SE, S, SW, W). 

\begin{figure}[!h]
  \includegraphics[width=\columnwidth]{whycontinuity.png}
  \caption{Continuity is implicit in choice of visualization rather than explicitly in choice of data container. The line plots in (b) are generated by a 2D table (a). Structurally this table can be identical to the 2D matrix (a) that generates the image in (c).}
  \label{fig:related-work:visual-algorithm}
\end{figure}

Often continuity is expressed in the choice of visual algorithm (visualization type), as explored in taxonomies by Tory and M\"{o}ller \cite{toryRethinkingVisualizationHighlevel2004} and Chi\cite{chiTaxonomyVisualizationTechniques2000}. For example, in \autoref{fig:related-work:visual-algorithm} the same table can be interpreted as a set of 1D continuous curves when visualized as a collection of line plots or as a 2D surface when visualized as an image.  This means that often there is no way to express data continuity independent of visualization type, meaning most visualization libraries will allow, for example, visualizing discrete data as a line plot or an image. General purpose visualization libraries-such as Matplotlib\cite{hunterMatplotlib2DGraphics2007}, Vtk\cite{hanwellVisualizationToolkitVTK2015,geveciVTK2012}, and D3 \cite{bostockDataDrivenDocuments2011}-carry distinct data models as part of the implementation of each visual algorithm. The lack of unified data model means that each plot in a linked\cite{beckerBrushingScatterplots1987,bujaInteractiveData1991} visualization is treated as independent, as are the transforms converting each field in the data to a visual equivalent.

Domain specific libraries can guarantee consistency because they have a single model of the data in their software design, as discussed in Heer and Agarwal \cite{HeerSoftware2006}'s survey of visualization software design patterns. For example, the relational database is core to tools influenced by APT, such as Tableau\cite{StoltePolaris2002,hanrahanVizQL2006,MackinlayShowme2007} and the Grammar of Graphics\cite{wilkinsonGrammarGraphics2005} inspired ggplot\cite{wickhamGgplot2ElegantGraphics2016a}, Vega\cite{satyanarayanDeclarativeInteractionDesign2014} and Altair\cite{vanderplasAltairInteractiveStatistical2018}. Images underpin scientific visualization tools such as Napari\cite{nicholas_sofroniew_2021_4533308} and ImageJ\cite{schneiderNIHImageImageJ2012} and the digital humanities oriented ImagePlot\cite{studiesCulturevisImageplot2021} macro; the need to visualize and manipulate graphs has spawned tools like Gephi\cite{bastianGephiOpenSource2009}, Graphviz\cite{ellsonGraphvizOpenSource2002}, and Networkx\cite{HagbergExploringNetwork2008}. 

 
\subsubsection{Fiber Bundles}
\label{sec:related-work:fiber-bundles}
\begin{figure}[h!]
  \includegraphics[width=\columnwidth]{fiberbundle.png}
  \caption{A fiber bundle is mathematical construct that allows us to express the relationship between data and continuity. The \textcolor{total}{total} space \dtotal is the topological space in which the data is embedded. The \textcolor{fiber}{fiber} space \dfiber\ is embedded in \dtotal\ and is the set of all possible values that any
  \note{add big rectangle E}}
  \label{fig:related-work:fiber-bundle}
\end{figure}

The model described in this work provides a method for expressing different types of continuities and nested continuities using the same model. We obtain this generality by using the mathematical theory of fiber bundles as the basis of our abstraction, as proposed by Butler, Bryson, and Pendley\cite{butlerVisualizationModelBased1989,butlerVectorBundleClassesForm1992}. In this paper, we build on their work that proposes using topological spaces to represent different properties of data. Here we present a brief summary of topology, for more information see Hatcher\cite{hatcherAlgebraicTopology2002}, Munkres\cite{munkresElementsAlgebraicTopology1984}, and Bradley et. al. \cite{bradleyTopologyCategoricalApproach2020}. 

A topological space a topological space $(X, \mathscr{T})$ is a set $X$ with a topology $\mathscr{T}$. Topologies are collections of open sets 


such that the empty set and $X$ are in the collection of open sets $\mathscr{T}$, the union of elements in $\mathscr{T}$


Specifically, Butler, Bryson, and Pendley suggest that fiber bundles be the basis of an abstract data model. Fiber bundles are a collection $(\dtotal, \dbase, \dfiber, \pi)$ of topological spaces

\begin{LaTeXdescription}
  \item[\textcolor{total}{Total Space} \dtotal]
  \item[\textcolor{fiber}{Fiber Space} \dfiber]
  \item[\textcolor{base}{Base Space} \dbase]  
\end{LaTeXdescription}

with a projection map $\pi:\dtotal\rightarrow\dbase$ that connects every point in \dtotal\ to a point in \dbase. 
\begin{equation}
  \label{eq:fiber_bundle}
  \begin{tikzcd}
      \dfiber \arrow[r, hook] & \dtotal \arrow[r, "\pi"] & \dbase
  \end{tikzcd}
\end{equation}

As indicated by $\hookrightarrow$, the fiber space \dfiber\ is embedded inside the total space \dtotal\. This is illustrated in \autoref{fig:related-work:fiber-bundle}, wherein values lives in the fiber $\dfiber\subseteq\dtotal$. In this example, the fiber \dfiber\ is the cartesian product of two sets $F_{0}\times F_{1}$ where each fiber $F_{i}$ is ....
\begin{equation}
  formal equation of the fiber?
  \label{eq:related-work:fiber}
\end{equation}

While the values are embedded in \dfiber, the continuity of the data is modeled as the base space \dbase, which is the quotient space \cite{QuotientSpaceTopology2020}
\begin{equation}
formal math of the base space as equivalence class for F_[k]
\end{equation}
which means that every point in $\dfiber_{k}$ maps to a point $k \subset K$. 

\begin{equation}
  \begin{tikzcd}
      \dfiber \arrow[r, hook] & \dtotal \arrow[d, "\pi"'] \\
                        & \dbase \arrow[u, "\dsection"', bend right]
  \end{tikzcd}
\end{equation}

\subsection{Sheaf Maps \sheaf}
We can use sheafs to ensure continuity even when the data is broken up. Is the glue rules -> we can haz parallism. 
\begin{equation}
  \vartist: \sheaf(\dtotal) \rightarrow \sheaf(\gtotal)
  \label{eq:math:artist:artist}
\end{equation}

\begin{figure}[h!]
  \includegraphics[width=\columnwidth]{sheaf.png}
  \caption{}
\end{figure}


\subsection{Equivariance}
Many visualization theorists have expressed the notion that scales, which are maps from data to visual representation, 
Hullman and Qu scales \cite{hullmanKeeping2018}

\begin{figure}[!h]
  \includegraphics[width=\columnwidth]{equiv.png}
  \caption{Equivariance is that a transformation on the data has a corresponding transformation in the graphical representation. For example, in this figure the data is scaled by a factor 10. Equivalently the line plot is scaled by factor of 10, resulting in a shrunken line plot. Either a transformation on the data side can induce a transformation on the visual side, or a transformation on the visual side indicates that there is also a transformation on the data side. }
\end{figure}

\begin{tikzcd}
  data \arrow[r] \arrow[d, "function"] & representation \arrow[r] & visual\;stimulus \arrow[d, "visual\;equivalent\;to\;function"] \\
  data \arrow[r]                       & representation \arrow[r] & visual\; stimulus                                          
\end{tikzcd}

\subsubsection{Category Theory}
In this work, we propose that equivariance constraints can be expressed using category theory. Vickers et. al provide a brief introduction to category theory for visualization practitioners \cite{vickersUnderstandingVisualizationFormal2013}, but their work focuses on data, representation, and evocation, while this paper is aims to provide guidance on how the map from data to representation should be implemented. 

\section{Artist} 
%%- brief intro to artist in most simple form $\vartist \dtotal \rightarrow \gtotal$ 
The \textcolor{artist}{Artist $\mathcal{\vartist}$} is a transformation from a 
%% flesh out categorical framing of artist, walk through how eq 16 is part of implementing the artists
%% https://github.com/story645/proposal/blob/main/notes/meetings/2021_08_30.md

%% domain of A is category E, with sheaves on E7   
%% range of A is category H, sheaves on H
%% category bundles, functions are bundles maps, 
%% subset/restriction is a type of bundle map - bundle over subset is bundle over whole thing, which induces map on section which goes other way which is restriction
%% bundle map induces a sheaf map, fibers themselves are categories, map from fiber to fiber that's legit in F
%%% what is the set up? category of bundles, bundle maps - fiber have some structure/any structure/no structure - bundle maps should be functors, A is gonna be a functor, H is a category of bundles + sheaves of bundles, A is a functor
%%% we claim that for visualizations, A decomposes as 
%%% nu bundle map on bundle + functor on fibers, Q takes you to H
%%% K to S is pullback of deform retract, A is functor on imageA in H

We propose that the artist $\mathcal{\vartist}$ is a morphism of presheaves defined to be a natural transformation (\autoref{sec:related-work:category-theory})
\note{ommutative diagram maybe?}

We can then explain inclusion & continuity as special cases, preservation as consequence of this. Generalize inclusion of open sets and restriction (how topology is put together) and equivariance on the fiber in terms of natural transformations <- condition we want to satisfy is natural transformation. Constraint on artist is that it's a natural transformations, will construct that way, will show 

1) bundle E restricted to U is presheaf, 

F_1: \mathcal{K_1} \rightarrow Set <-presheaf is the functor F_1
any map on base spaces will give us a map on functors, sections will pull back

artist w. particular base space w/ particular, 

How do you define the category E
objects: F_1 
  - presheaf: is it defined on fixed K and E0 (this provides the Equivariance on fibers + continuity), declaring it as E->V is natural transformation
  - presheaves: is it defined on arbitrary K and Eo
morphisms: inclusion -> these go back and forth, hence is a profunctor

fiber preserving maps 
fiber as category -> map preserves the structure, monotonic etc

How are you defining the artist? fix E0 and K and nu and q

same sheaf w/ arrow to itself , requires that all bundles be morphisms of a bundle to itself
  - morphisms are actions, bundle to bundle which when restricted to bundle are morphisms of bundle to itself 

- presheaf is just a functor

F is object in category that just has itself as only object 
if we have a bundle transfrom F_0 to F_0, then we have a natural transfrom F_0 to F_0

same thing has to happen on V side, declare which bundle maps are OK on which side, nu goes from one to the other, preserving the structure3 , doesn't have to have the same dimension,
- we don't care about a change in fiber dim - nu defines type it eats, which is static and defined relative to the fibers - can't change K, or E, nu is fixed on F/E, V and E can be different dimensions

category w/ one functor E_0, morphisms are functor to itself, f is data to data, g is graphic to graphic, nu(f) \rightarrow (g), when we get to H bundle, 
V over K -> H over S, 

nu is bundle to bundle (natural transformation)
Q is a natural transformation - proposition, map that satisifies this condition, v \rightarrow h v, 

\begin{equation}
  \vartist: \dtotal \rightarrow \gtotal
  \label{eq:math:artist:artist}
\end{equation}

\begin{definition} A presheaf is a category with extra structure - definition
  
\end{definition}

\begin{definition} a morphism of presheaves is defined to be a natural transform
  
\end{definition}

1) start w/ presheaf definition and explain K etc in context, $\Gamma(U, E)$ is an object in set, presheaf(U)->\Gamma(E,U)


\begin{definition} 
  \begin{enumerate}
    \item \textit{objects} all open sets $U_{i} \in \{U\}$ in \dbase, including the empty set $\varnothing$ and the maximum set \dbase
    \item \textit{morphisms} inclusion maps $\iota: U_{i} \hookrightarrow U_{j}$ 
  \end{enumerate} 
\end{definition}

\begin{definition} The category $\mathcal{\E}$ consists of 
  \begin{enumerate}
    \item\textit{objects} bundles \dtotal
    \item\textit{morphisms} inclusion maps: $\iota: E_i \hookrightarrow E_j$
  \end{enumerate}
\end{definition}

\begin{definition} 
\end{definition}

\subsection{Data}
We model data as sections of a fiber bundle $(\dtotal, \pi, \dbase, \dfiber)$. We encode the continuity of the data as the \textcolor{base}{base space} $\dbase$. \note{does this go in intro? This is mostly important for data, and we kinda use it for visual intermediary but not really for H} The \textcolor{fiber}{fiber space} \dfiber\ is the space of all possible values of the data \autoref{eq:related-work:fiber}. To allow us to map multiple fields of the same datatype to distinct fibers, we adopt Spivak's formulation of the fiber as a (column name, data domain) simple schema \cite{spivakSIMPLICIALDATABASES,spivakDatabasesAreCategories2010}. Spivak models the fiber space as a fiber bundle with sections $\sigma:C \rightarrow DT$ that take as input column name $c \in C$ and returns as output the set of values associated with the type of the field $T \in DT$, for example $\mathbb{R}$ for \texttt{floats}. In the paper, the $\dfiber$ is the cartesian cross product of these datatype domains \textbf{DT}. We model the data as the section \dsection because, as described in \autoref{sec:related-work:fiber-bundles}, the section is the map from the indexing space \dbase to the space of possible data values \dfiber. Spivak's notation also allows for associating elements in a section with the field it comes from. This allows for field based selection of values, while inclusion allows for continuity (index) based selections. 

\begin{figure}[h!]
  \includegraphics[width=\columnwidth]{fiberbundle.png}
  \caption{\note{replace with more concrete}}
  \label{fig:artist:data}
\end{figure}

One example of encoding data as a section of a fiber bundle is illustrated in \autoref{fig:artist:data}. In this example, the data is a \note{not totally decided yet} table of weather station data. Here we are interested in the time series of temperature values in the data, so we encode the continuity as the 1D interval \dbase. In this multivariate data set, the fields we want to visualize are \texttt{time}, \texttt{temperature}, and \texttt{station}. The fiber space \dfiber\ is the cartesian cross product of the fibers of each field
\begin{equation*}
  \dfiber = \dfiber_{time} \times \dfiber_{temperature} \times \dfiber_{station}
\end{equation*}
where each field fiber is the set of values that are valid for the field: 
\begin{align*}
  \dfiber_{time} &= \mathbb{R} \\
  \dfiber_{temperature} &= \mathbb{R}\\
  \dfiber_{station} &= \{s_0, s_1, \cdots, s_i, \cdots, s_n\} 
\end{align*}

The section \dsection\ is the abstraction of the data being visualized. The section at a point $k \in K$ in the base space returns a value from each field in the fiber.
\begin{equation*}
  \tau(k) = ((time, t_k); (precipitation, p_k); (station\;name, n_k))
\end{equation*}


\subsection{Graphic} 
The object of the graphic category $\mathcal{\gtotal}$ is the fiberbundle \gtotal. The bundle \gtotal\ has the same structure as the data bundle \dtotal

\begin{equation}
  \begin{tikzcd}[ampersand replacement=\&]
      \gfiber \arrow[r, hook] \& \gtotal \arrow[d, "\pi"'] \\
                        \& \gbase \arrow[u, "\gsection"', bend right]
  \end{tikzcd}
\end{equation}
with a fiber space \gfiber\ embedded in the total space \gtotal\ and a section map $\gsection:\gbase\rightarrow\gtotal$. The attributes of the graphic bundle $(\gtotal, \pi, \gfiber, \gbase)$ encode attributes of the graphic and display space 

\begin{LaTeXdescription}
\item [\textcolor{base}{base space} \gbase] continuity of display space (e.g. screen, 3D print)
\item [\textcolor{fiber}{fiber space} \gfiber] attributes of the display space (e.g a pixel = (x,y,r,g,b,a))
\end{Latex}
\item [\textcolor{section}{section} \gsection] graphic generating function
\end{LaTeXdescription}.

We use a fiber bundle because it allows us to generalize implementation details. In this work, \gtotal\ assumes the the display is an idealized 2D screen and \gsection\ is an abstraction of rendering. For example, \gsection\ can be a specification such as PDF\cite{bienz1993portable}, SVG\cite{quintScalable2003} or an OpenGL scene graph\cite{CarsonOpenGL1997}, or a rendering engine such as Cairo\cite{CairographicsOrg} or AGG\cite{shemanarevAntiGrainGeometry}.

\begin{figure}[h!]
  \includegraphics[width=\columnwidth]{render.png}
  \caption{}
  \label{fig:artist:graphic}
\end{figure}

\begin{example}
As illustrated in \autoref{fig:artist:graphic}, 
\end{example}


\begin{definition}[Category $\mathcal{\gtotal}$]
  
\end{definition}


\subsection{Union of Artists}
\label{sec:artist:union}
\begin{figure}[!h]
\centering
\subfloat[Union of H]{\includegraphics[width=.4\columnwidth]{exploding_artist.png}%
\label{fig_first_case}}
%\hfil
\subfloat[Artist with shared/harmonized \vsection]{\includegraphics[width=.4\columnwidth]{combined_artist.png}%
\label{fig_second_case}}
\caption{Simulation results for the network.}
\label{fig_sim}
\end{figure}


\section{Constraints: Construction and formal properties of Artists}
\begin{figure*}[h!]
  \includegraphics[width=\linewidth]{q.png}
  \caption{}
  \label{fig:constraints:q-overall}
\end{figure*}


\begin{definition} Following from \label{def:artist:} the artist \vartist is a special case of  natural transformations
  \begin{equation}
    \vartist: \Gamma(\dtotal, \dbase)\nrightarrow \Gamma(\gtotal, \gbase)
  \end{equation}
  between the profunctor objects $\Gamma(E),\Gamma(H)$ that preserves continuity and equivariance.
\end{definition}
  Natural transformations are functions that map functions between categories (functors) to other functors in a structure preserving way \cite{riehlCategoryTheoryContext, spanier1989algebraic, fongInvitationAppliedCategory2019}.

\begin{definition} We define the \textcolor{artist}{Artist} natural transformation \vartist\ as the tuple $(\vindex, \vchannel, \vmark, \dtotal, \vtotal, \gtotal)$ where
  \begin{enumerate}
    \item \vindex\ is a continuity preserving functor
    \item \vchannel\ and \vmark\ are equivariance preserving functors
    \item \dtotal, \vtotal, and \gtotal\ are profunctor categories of sheafs (\autoref{def:artist:profunctor})
  \end{enumerate}
\end{definition}


\begin{equation}
  \label{eq:math:artist:diagram}
  \begin{tikzcd}
      \dtotal \arrow[r, "\vchannel"] \arrow[rd, "\pi"'] & \vtotal \arrow[d, "\pi"] & \vindex^*\vtotal \arrow[r, "\vmark"] \arrow[d, "\vindex^*\pi"'] \arrow[l, "\vindex^*"'] & \gtotal \arrow[ld, "\pi"] \\
                                            & \dbase                  & \gbase \arrow[l, "\vindex"']                                              &                    
      \end{tikzcd}
\end{equation}

\subsection{Graphic to Data: \vindex}
\begin{equation}
  \begin{tikzcd}
      \dtotal \arrow[d, "\pi"'] & \gtotal \arrow[d, "\pi"'] \\
      \dbase                   & \gbase \arrow[l, "\vindex"']
  \end{tikzcd}
  \label{eq:math:graphic:vindex}
\end{equation}


\subsection{Visual Bundle \vtotal}
\begin{equation}
  \begin{tikzcd}[ampersand replacement=\&]
      \vfiber \arrow[r, hook] \& \vtotal \arrow[d, "\pi"'] \\
                        \& \dbase \arrow[u, "\vsection"', bend right]
  \end{tikzcd}
\end{equation}

\subsection{Data to Visual Encodings: \vchannel} %look at what K&S call these stages
\note{needs a sentence on why this is broader than scales}
\begin{figure}[!h]
  \centering
  \subfloat[Artists with shared $\mu_i$ renderered correctly]{\includegraphics[width=1.2in]{partial_fixed.png}%
  \label{fig_first_case}}
  %\hfil
  \subfloat[Artists without shared $\mu_i$]{\includegraphics[width=1.2in]{partial_invalid.png}%
  \label{fig_second_case}}
  \caption{Simulation results for the network.}
  \label{fig_sim}
  \end{figure}

\begin{equation}
  \label{eq:math:artist:nu}
  \{\vchannel_{0}, \ldots, \vchannel_{n}\}: \{\dsection_{0}, \ldots, \dsection_{n}\} \mapsto \{\vsection_{0}, \ldots, \vsection_{n}\}
\end{equation}

We enforce the equivariance constraint

\begin{equation}
  \label{eq:math:artist:nu_commute}
\begin{tikzcd}
  \dtotal_i \arrow[r] \arrow[r, "\vchannel_i"] \arrow[d, "m_{\delement}"'] & \vtotal_i \arrow[d, "m_{\velement}"] \\
  \dtotal_i \arrow[r, "\vchannel_i"]                           & \vtotal_i               
\end{tikzcd}
\end{equation}

\subsubsection{Visual to Graphic: \vmark} % look at what K&S call these stages
Visual encodings to something like marks 
\begin{figure}[!h]
  \includegraphics[width=\columnwidth]{diff_type_q.png}
  \caption{rework this as a commutative box w/ the r in E row associated w/ this qhat(k)}
\end{figure}
 
\section{Case Study}
\begin{figure}[h!]
  \includegraphics[width=\columnwidth]{path_of_q.png}
    \caption{\note{add in xi!}}
  \label{fig:api}
\end{figure}
We implement the \textbf{arrows} in \autoref{fig:api}. \mintinline{python}{axesArtist} is a parent artist that acts as a screen. This allows for the composition described in \autoref{sec:artist:union}

\subsection{\vartist}
\begin{minted}{python}
for local_tau in axesArtist.artist.data.query(screen_bounds, dpi):
    mu = axesArtist.artist.graphic.mu(local_tau)
    rho = axesArtist.artist.graphic.qhat(**mu)
    H = rho(renderer)
\end{minted}

where the artist is already parameterized with the \vindex\ functions and which fibers they are associated to:

\begin{min


\subsubsection{\vindex}
\subsubsection{\vchannel}
\subsubsection{\vmarkd}



\section{Discussion}
\subsection{Limitations}
\subsection{future work}

\section{Conclusion}
The conclusion goes here.


\appendices
\section{Rendering: \gsection}
\section{Manufacturing $\vmarkd \leftarrow \vmark$}
\begin{equation}
  \begin{tikzcd}
      \textcolor{gray!50}{\dtotal} \arrow[r, "\vchannel", color=gray!50] \arrow[rd, "\pi"', color=gray!50] & \vtotal \arrow[d, "\pi"']                  & \vindex^*\vtotal \arrow[r, "\vmark", color=gray!50] \arrow[d, "\vindex^*\pi"'] \arrow[l,  "\vindex^*"'] & \textcolor{gray!50}{\gtotal} \arrow[ld, "\pi", color=gray!50] \\ & \dbase \arrow[u, "\vsection"', bend right] & \gbase \arrow[l, "\vindex"'] \arrow[u, "\vindex^*\vsection"', bend right]               &                          
      \end{tikzcd}
      \label{eq:math:artist:qhat}
\end{equation}



% use section* for acknowledgment
\ifCLASSOPTIONcompsoc
  % The Computer Society usually uses the plural form
  \section*{Acknowledgments}
\else
  % regular IEEE prefers the singular form
  \section*{Acknowledgment}
\fi


The authors would like to thank...


% Can use something like this to put references on a page
% by themselves when using endfloat and the captionsoff option.
\ifCLASSOPTIONcaptionsoff
  \newpage
\fi

% trigger a \newpage just before the given reference
% number - used to balance the columns on the last page
% adjust value as needed - may need to be readjusted if
% the document is modified later
%\IEEEtriggeratref{8}
% The "triggered" command can be changed if desired:
%\IEEEtriggercmd{\enlargethispage{-5in}}

% references section

% can use a bibliography generated by BibTeX as a .bbl file
% BibTeX documentation can be easily obtained at:
% http://mirror.ctan.org/biblio/bibtex/contrib/doc/
% The IEEEtran BibTeX style support page is at:
% http://www.michaelshell.org/tex/ieeetran/bibtex/
\bibliographystyle{IEEEtran}
% argument is your BibTeX string definitions and bibliography database(s)
\bibliography{bibliography}

% biography section 
% If you have an EPS/PDF photo (graphicx package needed) extra braces are
% needed around the contents of the optional argument to biography to prevent
% the LaTeX parser from getting confused when it sees the complicated
% \includegraphics command within an optional argument. (You could create
% your own custom macro containing the \includegraphics command to make things
% simpler here.)
%\begin{IEEEbiography}[{\includegraphics[width=1in,height=1.25in,clip,keepaspectratio]{mshell}}]{Michael Shell}
% or if you just want to reserve a space for a photo:

%\begin{IEEEbiography}{Michael Shell}
%\end{IEEEbiography}

% if you will not have a photo at all:
\begin{IEEEbiographynophoto}{Hannah Aizenman}
Biography text here.
\end{IEEEbiographynophoto}

\begin{IEEEbiographynophoto}{Thomas Caswell}
  Biography text here.
\end{IEEEbiographynophoto}
% insert where needed to balance the two columns on the last page with
% biographies
%\newpage

\begin{IEEEbiographynophoto}{Michael Grossberg}
Biography text here.
\end{IEEEbiographynophoto}

% You can push biographies down or up by placing
% a \vfill before or after them. The appropriate
% use of \vfill depends on what kind of text is
% on the last page and whether or not the columns
% are being equalized.

%\vfill

% Can be used to pull up biographies so that the bottom of the last one
% is flush with the other column.
%\enlargethispage{-5in}

% that's all folks
\end{document}


